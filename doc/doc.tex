%        File: doc.tex
%     Created: Sat Oct 26 03:00 AM 2013 E
% Last Change: Sat Oct 26 03:00 AM 2013 E
%
\documentclass[a4paper,12pt]{report}
\usepackage[margin=3cm]{geometry}
\usepackage[finnish]{babel}
\usepackage[utf8]{inputenc}
\usepackage[pdftex]{graphicx}
\usepackage{setspace}
\onehalfspacing
\DeclareGraphicsRule{*}{mps}{*}{}
\title{tsoha\\Kalenterisovellus}
\author{Samuli Thomasson}
\begin{document}

% CryptoJS !

\maketitle
\tableofcontents
\chapter{Johdanto}
Tämä kalenterijärjestelmä mahdollistaa käyttäjille kalenterien ylläpidon
selain\-käyttö\-liittymästä.  Sovelluksen käyttäjä pystyy katsomaan, lisäämään
ja muokkaamaan (kalenteri)merkintöjä omissa kalentereissaan.  Kalenteri koostuu
joukosta merkintöjä, kuten tapahtumia ja muistutuksia.  Järjestelmä mahdollistaa
merkintöjen asettamisen julkisesti nähtäville. Myös kokonaisista kalentereista
voi tehdä julkisia. Julkisesta kalenterista voi myös tehdä vapaasti muokattavan,
jolloin kuka tahansa voi ehdottaa merkintöjä kalenteriin. Vapaasti muokattavassa
kalenterissa lopullinen määräysvalta on kalenterin omistajalla -- omistajan omia
tai hyväksymiä merkintöjä voi poistaa vain omistaja.

Järjestelmän toteutetuskieli on Haskell ja kehyksenä toimii \emph{Yesod}.
Järjestelmä on yksinkertaisuudessaan ohjelma, joka kuuntelee jossain portissa
web-palvelimena. Sitä voi siis ajaa itsenäisenä web-palvelimena suoraan portissa
80. Yleensä samankaltaiset sovellukset kuitenkin ajetaan paikallisessa portissa
ja välitetään dedikoidun web-palvelimen (Nginx ja vastaavat) kautta.

Järjestelmän alustajärjestelmältä vaaditaan, että lähdekoodi voidaan kääntää
sille. Staattisten kirjastojen ansoista ei käyttöönotto vaadi alustalta kuin
käännetyn ohjelman ajamisen, staattiset verkkosivuresurssit (JavaScript ja CSS)
sekä tuetun tietokannan. Ohjelma pitää ajaa omassa portissaan, eli ohjelman
käyttö vaatii joko ainoan web-palvelimen aseman ulkoisessa portissa, tai
(parempi) käänteisproxyä ymmärtävän web-palvelimen alustalta.

Järjestelmä suunnitellaan PostgreSQL-kannalle; vaihdon edellytyksiä on että
persistent tukee sitä, ja että tekijä ei ole käyttänyt liian erikoisia
SQL-konstruktioita.  Itse järjestelmän ominaisuuksien käyttöön asiakasohjelmassa
ei tarvitse JavaScriptiä. JavaScriptiä voidaan kuitenkiin käyttää
käyttökokemuksen parantamiseksi.

%%%%%%%%%%%%%%%%%%%%%%%%%%%%%%%%%%%%%%%%%%%%%%%%%%%%%%%%%%
\chapter{Järjestelmä}

\section{Yleiskuva järjestelmästä}
Käyttäjät toimivat aina selainkäyttöliittymän kautta. Käyttätapauskaaviossa
(\ref{graph_usecases}) on esitelty ohjelman päätoiminnot. Ne on jaettu kolmeen
pääryhmään: käyttäjien autentikointi, operaatiot omassa kalenterissa ja
operaatiot jonkun muun kalenterissa.

\begin{figure}[h]
   \centering \includegraphics{usecases.1}
   \caption{Kalenterijärjestelmän käyttötapauskaavio.}
   \label{graph_usecases}
\end{figure}

Järjestelmän käyttäjiä on kahdenlaisia: vierailijoita ja varsinaisia käyttäjiä:
\begin{description}
   \item[Vierailijoita] ovat kaikki jotka eivät ole kirjautunu sisään.
   \item[Käyttäjiä] ovat kaikki sisäänkirjautuneet.
\end{description}

\section{Käyttötapaukset}
\subsection{Autentikointi}
\subsection{Oma kalenteri}
\subsection{Vieras kalenteri}
\section{Järjestelmän tietosisältö} <++>
\section{Relaatiotietokantakaavio} <++>
\section{Järjestelmän yleisrakenne} <++>
\section{Järjestelmän komponentit} <++>
\section{Käyttöliittymä} <++>

%%%%%%%%%%%%%%%%%%%%%%%%%%%%%%%%%%%%%%%%%%%%%%%%%%%%%%%%%%
\chapter{Käyttö}
\section{Asennustiedot} <++>
\section{Käynnistys- / käyttöohje} <++>

%%%%%%%%%%%%%%%%%%%%%%%%%%%%%%%%%%%%%%%%%%%%%%%%%%%%%%%%%%
\chapter{Testaus, tunnetut bugit ja puutteet \& jatkokehitysideat} <++>

jatko: ryhmiä, icalendar,
\bibliography{refs.bib}

\end{document}


