%        File: doc.tex
%     Created: Sat Oct 26 03:00 AM 2013 E
% Last Change: Sat Oct 26 03:00 AM 2013 E
%
\documentclass[a4paper,12pt]{report}
\usepackage[left=3cm,right=3cm,top=2cm,bottom=2cm]{geometry}
\usepackage[finnish]{babel}
\usepackage[utf8]{inputenc}
\usepackage{setspace}\onehalfspacing
\usepackage[pdftex]{graphicx}
\DeclareGraphicsRule{*}{mps}{*}{} % meta-uml

\title{tsoha\\Kalenterisovellus}
\author{Samuli Thomasson}
\begin{document}

% Extra!! CryptoJS?

\maketitle
\tableofcontents
\chapter{Johdanto}
Tämä kalenterijärjestelmä mahdollistaa käyttäjille kalenterien yllä\-pidon
selaimesta.  Sovelluksen käyttäjä pystyy katsomaan, lisäämään
ja muokkaamaan (kalenteri)merkintöjä omissa kalentereissaan.  Kalenteri koostuu
joukosta merkintöjä, kuten tapahtumia ja muistutuksia.  Järjestelmä mahdollistaa
merkintöjen asettamisen julkisesti nähtäville. Myös kokonaisista kalentereista
voi tehdä julkisia. Julkisesta kalenterista voi myös tehdä vapaasti muokattavan,
jolloin kuka tahansa voi ehdottaa merkintöjä kalenteriin. Vapaasti muokattavassa
kalenterissa lopullinen määräysvalta on kalenterin omistajalla -- omistajan omia
tai hyväksymiä merkintöjä voi poistaa vain omistaja.

Järjestelmän toteutetuskieli on Haskell ja kehyksenä toimii \emph{Yesod}.
Järjestelmä on yksinkertaisuudessaan ohjelma, joka kuuntelee jossain portissa
web-palvelimena. Sitä voi siis ajaa itsenäisenä web-palvelimena suoraan portissa
80. Yleensä samankaltaiset sovellukset kuitenkin ajetaan paikallisessa portissa
ja välitetään dedikoidun web-palvelimen (Nginx ja vastaavat) kautta.

Järjestelmän alustajärjestelmältä vaaditaan, että lähdekoodi voidaan kääntää
sille. Staattisten kirjastojen ansoista ei käyttöönotto vaadi alustalta kuin
käännetyn ohjelman ajamisen, staattiset verkkosivuresurssit (JavaScript ja CSS)
sekä tuetun tietokannan. Ohjelma pitää ajaa omassa portissaan, eli ohjelman
käyttö vaatii joko ainoan web-palvelimen aseman ulkoisessa portissa, tai
(parempi) käänteisproxyä ymmärtävän web-palvelimen alustalta.

Järjestelmä suunnitellaan PostgreSQL-kannalle; vaihdon edellytyksiä on että
persistent tukee sitä, ja että tekijä ei ole käyttänyt liian erikoisia
SQL-konstruktioita.  Itse järjestelmän ominaisuuksien käyttöön asiakasohjelmassa
ei tarvitse JavaScriptiä. JavaScriptiä voidaan kuitenkiin käyttää
käyttökokemuksen parantamiseksi.

%%%%%%%%%%%%%%%%%%%%%%%%%%%%%%%%%%%%%%%%%%%%%%%%%%%%%%%%%%
\chapter{Järjestelmä}

\section{Yleiskuva järjestelmästä}
Käyttäjät toimivat aina selainkäyttöliittymän kautta. Käyttätapauskaaviossa
(\ref{graph_usecases}) on esitelty ohjelman päätoiminnot. Ne on jaettu kolmeen
pääryhmään: käyttäjien autentikointi, operaatiot omassa kalenterissa ja
operaatiot jonkun muun kalenterissa.

Järjestelmän käyttäjiä on kahdenlaisia: vierailijoita ja varsinaisia
käyttäjiä:
\begin{description}
   \item[Vierailijoita] ovat kaikki jotka eivät ole kirjautuneet sisään,
      riippumatta siitä onko heillä tunnisteita järjestelmässä vai ei.
   \item[Käyttäjiä] ovat kaikki sisäänkirjautuneet.
\end{description}

\section{Käyttötapaukset}
Tässä luvussa esitellään sovelluksen eri käyttötapaukset.  Ellei toisin mainita,
niin esiteltävät käyttö\-tapaukset ovat vain varsinaisten käyttäjien saatavissa.
Käyttö\-tapaukset on jaettu kolmeen joukkoon: autentikointi, kalenterit ja
vieraskalenterit.  Käyttö\-tapaus\-kaavio (\ref{graph_usecases}) havainnollistaa
tapauksia; eri kategoriat on siinä jaettu visaalisesti lohkoihin.
\begin{figure}[ht]
   \centering \includegraphics{usecases.1}
   \caption{Kalenterijärjestelmän käyttötapauskaavio.}
   \label{graph_usecases}
\end{figure}

\subsection{Autentikointi}
\begin{description}
   \item[Rekisteröityminen]  \emph{Vain vieraille}. Ensimmäinen asia jonka uuden
      tulevan käyt\-täjän täytyy valitettavasti tehdä, on rekisteröidä joitain
      tunnisteita itsestään järjes-telmään.  Näihin tietoihin täytyy lukeutua
      jotain, millä käyttäjä voi todentaa itsensä myöhemmin olevan sama
      käyttäjä.

      Rekisteröitymiseen tarjotaan monia vaihtoehtoja. Perinteinen käyttäjänimen
      ja salasanan tallennus palvelimelle on vakiovaihtoehto, mutta myös
      moderneja metodeja on tarjolla: BrowserId, OpenId ja Gmail. 

   \item[Sisäänkirjautuminen]  \emph{Vain vieraille}. Käyttäjien täytyy pystyä
      autentikoimaan itsensä rekiste\-röityneeksi käyttäjäksi (mahdollisesti
      itsekseen), eli korottamaan itsensä vieraasta käyttäjäksi järjestelmän
      kontekstissa.  Sisäänkirjautumiseen käy\-tetään samaa metodia jota
      käyttäjä käytti rekisteröitymiseen.
\end{description}

\subsection{Kalenterit}\label{oma_kalenteri}
Käyttäjällä voi olla usea kalenteri järjestelmässä.  Käyttäjän luomaa kalenteria
kutsutaan \emph{käyttäjän omistamaksi} kalenteriksi.  Olemassaolevan kalenterin
omistajaa ei voi vaihtaa.

Kalenterioperaatiot:
\begin{description}
   \item[Uusi kalenteri] Kalenterin luominen on ensimmäinen asia jonka uusi
      käyttäjä tekee, ja se on helppoa: häneltä kysytään heti ensimmäisen
      sisäänkirjautumisen jälkeen nimeä kalenterille tekstikentällä.
      Kalenterille voi halutessaan asettaa jonkin toisen väriattribuutin kuin
      oletuksen.  Lisää attribuutteja voi varsinaisesti säätää myöhemmin, kun
      kalenteri on luotu.
   \item[Kalenterinäkymä]
      Kalenterien tarkastelu on koko järjestelmän tärkein ja laajin
      käyttötapaus.  Sen voisi esittää myös useana pienempänä käyttötapauksena.

      Kalentereita ei pääasiassa tarkastella yksitellen, vaan useita
      päällekkäin.  Tarkastelu tapahtuu nk.  \emph{kalenterinäkymässä}.
      Kalenterinäkymässä näytetään kerrallaan yhden viikon tapahtumat
      kalentereista\footnote{%
         Jos aikaa riittää niin näytettävä sisältö on paremmin säädettävissä
         (kuukausi, päivä, tai muita näkymiä).
      }.  Näkyvää viikkoa voi vaihtaa seuraavaan, edelliseen, senhetkiseen
      viikkoon tai johonkin tiettyyn viikkoon.

      Näkymässä näkyvät kalenterit voi valita.

      Näkymässä on esillä keinoja kalenterikohteiden operointiin.  Näkyvistä
      kalenterikohteista näytetään lyhyet yhteenvedot sekä keinot niiden
      muokkaukseen.
   \item[Kalenteritietojen muokkaus] Kalenterit ovat vain kohteita kokoavia
      \textbf{---insert-a-very-intelligent-word-here---}, mutta niillä on silti
      muutama muokattava attribuutti, jotka vaikuttavat niiden toimintaan. Näitä
      ovat:
      \begin{description}
         \item[Nimi]  Kalenterin muista identifioiva tunniste.
         \item[Väri]  Minkä värisinä kalenterin kohteet näkyvät
            kalenterinäkymässä.
         \item[Julkinen]  Voiko kalenterin kohteita tarkastella kuka vain
            (käyttäjä tai vieras).
         \item[Yhteinen]  Voiko kuka vain lisätä kalenteriin omia kohteita.
            Selvästi tämä attribuutti on järkevä vain kun kalenteri on
            julkinen.  
      \end{description}

      Kaikkia attribuutteja voi muokata muokkaus\-näkymässä, jossa näkyvät myös
      nykyiset arvot.
   \item[Poisto]  Kalenterin voi myös poistaa, ja sekin tapahtuu
      muokkausnäkymässä.  Poistoon kysytään varmistus mikäli kalenteri sisältää
      kohteita.
\end{description}


\subsubsection{Kohteet}
Kalenterit koostuvat erilaisista (kalenteri)kohteista.  Kaikille erilaisista
kohteista tarjotaan CRUD-rajapinnat, joissa keskinäiset erot muodostuvat
kohteiden eroavista tietokentistä.  Yhteisiä ominaisuuksia, kuten päivämääriä ja
toistuvuuksia, operoidaan samoin kohdevariantista riippumatta.

Yleiseen kalenterikohteeseen liittyvät käyttötapaukset:
\begin{description}
   \item[Luonti] lisays.
   \item[Tarkastelu]
   \item[Muokkaus]
   \item[Poisto]
   \item[Jako]
\end{description}

\subsection{Vieraskalenterit}
Vieraskalenteri-käyttötapaukset koskevat käyttäjiä, vieraita ja niitä
kalentereita jotka eivät ole heidän omistamia.  Vieraskalenteria pystyy
tarkastelemaan ja mahdollisesti (omistajan niin asettaessa) tekemään lisäyksiä,
mutta \emph{ei} muutoksia kalenteriin tai siinä oleviin muiden omistamiin
kalenterikohteisiin.  Näiden käyttötapausten oletus on, että kohteena olevan
kalenterin omistaja on sallinut julkiset toiminnot.  Käyttötapaukset:
\begin{description}
   \item[Tarkastelu] Vieraskalenteria voi tarkastella samalla tavalla kuin
      omaakin kalenteria (ks.~\ref{oma_kalenteri}).
   \item[Kohteen lisäys]  Vieraskalenteriin voi lisätä kohteita jos kalenterin
      omistaja on sen sallinut.  Lisäys tapahtuu samalla tavalla kuin oman
      kalenterin tapauksessa. 

      Vieraalta saatetaan vaatia kuvavarmennus.
\end{description}

%\section{Järjestelmän tietosisältö} <++>
%\section{Relaatiotietokantakaavio} <++>
%\section{Järjestelmän yleisrakenne} <++>
%\section{Järjestelmän komponentit} <++>
%\section{Käyttöliittymä} <++>

%%%%%%%%%%%%%%%%%%%%%%%%%%%%%%%%%%%%%%%%%%%%%%%%%%%%%%%%%%
%\chapter{Käyttö}
%\section{Asennustiedot} <++>
%\section{Käynnistys- / käyttöohje} <++>

%%%%%%%%%%%%%%%%%%%%%%%%%%%%%%%%%%%%%%%%%%%%%%%%%%%%%%%%%%
%\chapter{Testaus, tunnetut bugit ja puutteet \& jatkokehitysideat} <++>

% jatko: ryhmiä, icalendar,
\bibliography{refs.bib}

\end{document}


